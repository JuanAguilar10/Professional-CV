% Template:     Professional-CV
% Documento:    Funciones utilitarias
% Versión:      1.1.1 (02/10/2017)
% Codificación: UTF-8
%
% Autor: Pablo Pizarro R.
%        Facultad de Ciencias Físicas y Matemáticas
%        Universidad de Chile
%        pablo.pizarro@ing.uchile.cl, ppizarror.com
%
% Sitio web:    [http://latex.ppizarror.com/Professional-CV/]
% Licencia MIT: [https://opensource.org/licenses/MIT/]

\newcommand{\throwerror}[2]{
	% Lanza un mensaje de error
	% 	#1	Función del error
	%	#2	Mensaje
	\errmessage{LaTeX Error: \noexpand#1 #2 (linea \the\inputlineno)}
	\stop
}

\newcommand{\throwwarning}[1]{
	% Lanza un mensaje de advertencia
	%	#1	Mensaje
	\errmessage{LaTeX Warning: #1 (linea \the\inputlineno)}
}

\newcommand{\throwbadconfig}[3]{
	% Lanza un mensaje de error indicando mala configuración
	%	#1	Mensaje de error
	% 	#2	Configuración usada
	%	#3	Valores esperados
	\errmessage{LaTeX Warning: #1 \noexpand #2=#2. Valores esperados: #3}
	\stop
}

\newcommand{\throwbadconfigondoc}[3]{
	% Lanza un mensaje de error indicando mala configuración dentro de begin{document}
	%	#1	Mensaje de error
	% 	#2	Configuración usada
	%	#3	Valores esperados
	\errmessage{#1 \noexpand #2=#2. Valores esperados: #3}
	\stop
}

\newcommand{\checkvardefined}[1]{
	% Comprueba si una variable está definida
	%	#1	Variable
	\ifthenelse{\isundefined{#1}}{
		\errmessage{LaTeX Warning: Variable \noexpand#1 no definida}
		\stop
	}{}
}

\newcommand{\checkextravarexist}[2]{
	% Comprueba si una variable está definida
	%	#1	Variable
	%	#2	Mensaje
	\ifthenelse{\isundefined{#1}}{
		\errmessage{LaTeX Warning: Variable \noexpand#1 no definida}
		\ifx\hfuzz#2\hfuzz
			\errmessage{LaTeX Warning: Defina la variable en el bloque de INFORMACION DEL USUARIO al comienzo del archivo principal del Template}
		\else
			\errmessage{LaTeX Warning: #2}
		\fi
	}{}
}

\newcommand{\emptyvarerr}[3]{
	% Lanza un mensaje de error si una variable no ha sido definida
	% 	#1	Función del error
	%	#2	Variable
	%	#3	Mensaje
	\ifx\hfuzz#2\hfuzz
		\errmessage{LaTeX Warning: \noexpand#1 #3 (linea \the\inputlineno)}
	\fi
}

\newcommand{\setpagemargincm}[4]{
	% Cambia márgenes de las páginas [cm]
	% 	#1	Margen izquierdo
	%	#2	Margen superior
	%	#3	Margen derecho
	%	#4	Margen inferior
	\newgeometry{left=#1cm, top=#2cm, right=#3cm, bottom=#4cm}
}

\newcommand{\changemargin}[2]{
	% Cambia los márgenes del documento
	%	#1	Margen izquierdo
	%	#2	Margen derecho
	\emptyvarerr{\changemargin}{#1}{Margen izquierdo no definido}
	\emptyvarerr{\changemargin}{#2}{Margen derecho no definido}
	\list{}{\rightmargin#2\leftmargin#1}\item[]
}
\let\endchangemargin=\endlist

\newcommand{\maintitle}[3]{
	% Título principal con subtítulo y último cambio
	%	#1	Nombre autor
	%	#2	Edad corta autor
	%	#3	Último cambio
	\emptyvarerr{\maintitle}{#1}{Nombre autor no definido}
	\emptyvarerr{\maintitle}{#2}{Edad corta autor no definido}
	\emptyvarerr{\maintitle}{#3}{Fecha de ultimo cambio no definido}
	\noindent{\LARGE \textbf{#1}} \quad \emph{#2} \hfill \quad {\scriptsize \color{gray} #3} \\ \vspace{-1em}
}

\newcommand*{\roottitle}[1]{
	% Título de las secciones principales
	%	#1	Título
	\emptyvarerr{\roottitle}{#1}{Titulo no definido}
	\subsection*{#1}\vspace{-0.3em}\nopagebreak[4]
}

\newcommand*{\acr}[1]{
	% Crea un acrónimo de forma fácil
	%	#1	Acrónimo
	\textscale{.85}{#1}
}

\newcommand{\bodytext}[1]{
	% Crea un campo de texto
	% 	#1	Texto
	\nopagebreak[4]
	\begin{indentsection}
		\item[] #1
	\end{indentsection}
	\pagebreak[2]
}

\newcommand{\writeheader}{
	% Escribe el encabezado del CV
	
	% Escribe título
	\ifthenelse{\equal{\writetitleheader}{true}}{
		\begin{center}
			\fontsize{16}{12} \selectfont \vspace{0.3cm} \headertitle
		\end{center}
	}{}
	\ifthenelse{\equal{\writelastchangeheader}{true}}{
		\maintitle{\nombreautor}{\mesnacimientoautor \dianacimientoautor, \anonacimientoautor}{\nomlastchange\ \today}
	}{
		\maintitle{\nombreautor}{\mesnacimientoautor \dianacimientoautor, \anonacimientoautor}{}
	}
	
	% Escribe datos autor
	{
		\noindent \href{mailto:\correoautor}{\correoautor} \sbullet
		\telefonoautor \sbullet
		\href{\linkredsocial}{\linkredsocial} \\
		\direccionautor \sbullet \regionautor \sbullet \paisautor
	}
}

\newcommand{\personaltableentry}[2]{
	% Crea una entrada en la tabla de datos personales
	%	#1	Nombre entrada
	%	#2	Datos entrada
	\emptyvarerr{\personaltableentry}{#1}{Nombre entrada no definido}
	\emptyvarerr{\personaltableentry}{#2}{Datos entrada no definidos}
	\textbf{#1:} & #2 \\
}

\newcommand{\personaltablelastentry}[2]{
	% Crea una última entrada en la tabla de datos personales
	%	#1	Nombre entrada
	%	#2	Datos entrada
	\emptyvarerr{\personaltablelastentry}{#1}{Nombre entrada no definido}
	\emptyvarerr{\personaltablelastentry}{#2}{Datos entrada no definidos}
	\textbf{#1:} & #2 \vspace{-0.3cm}
}

\newcommand{\institutionentry}[4]{
	% Entrada en institución
	%	#1	Cargo
	%	#2	Fecha inicial
	%	#3	Fecha final
	%	#4	Descripción
	\emptyvarerr{\institutionentry}{#1}{Cargo o posición no definido}
	\emptyvarerr{\institutionentry}{#2}{Período no definido}
	\nopagebreak[4]
	\begin{indentsection}
		\item []
		\noindent
		\begin{minipage}{.7\linewidth}
			\textbf{#1}
		\end{minipage}
		\begin{minipage}{.29\linewidth}
			\begin{flushright}
				\noindent \emph{#2 -- #3}
			\end{flushright}
		\end{minipage}
		\ifx\hfuzz#4\hfuzz
			\begin{minipage}{1\linewidth}
				\vspace{0.2cm}
			\end{minipage}
		\else
			\begin{minipage}{1\linewidth}
				\vspace{0.2cm}
				\bodytext{#4}
				\par
			\end{minipage}
		\fi
	\end{indentsection}
	\nopagebreak[4]
}

\newcommand{\otherentry}[2]{
	% Otro tipo de entrada en cvblock
	%	#1	Título
	%	#2	Contenido
	\begin{basedescript}
		{\setlength{\leftmargin}{\doubleparindent}}
		\item[\hspace{\newparindent}\textbf{#1}] #2
	\end{basedescript}
	\vspace{-0.6cm}
}

\newcommand{\signatureentry}[1]{
	% Inserta un elemento en el bloque de firma
	%	#1	Elemento en la firma
	\texttt{\MakeUppercase{#1}} \\
}

% Crea un bull
\newcommand{\sbullet}{\ \ \raisebox{0.11em}[-1em][-1em]{\tiny $\bullet$} \ \ }

% Variaciones de vspace
\newcommand{\breakvspace}[1]{\pagebreak[2] \vspace{#1} \pagebreak[2]}
\newcommand{\nobreakvspace}[1]{\nopagebreak[4] \vspace{#1} \nopagebreak[4]}

% Línea espaciada
\newcommand{\spacedhrule}[2]{\breakvspace{#1} \hrule \nobreakvspace{#2}}

% Apóstrofe
\newcommand{\apo}{\raisebox{-.18ex}{'}{\hspace{-.1em}}}

% Evita salto de línea
\newcommand*{\nsp}{\hskip 0pt}
