% Template:     Professional-CV
% Documento:    Configuraciones del template
% Versión:      1.2.2 (08/10/2017)
% Codificación: UTF-8
%
% Autor: Pablo Pizarro R.
%        Facultad de Ciencias Físicas y Matemáticas
%        Universidad de Chile
%        pablo.pizarro@ing.uchile.cl, ppizarror.com
%
% Sitio web:    [http://latex.ppizarror.com/Professional-CV/]
% Licencia MIT: [https://opensource.org/licenses/MIT/]

% CONFIGURACIONES GENERALES
\def\documentlang {es-CL}         % Define el idioma del documento
\def\pointdecimal {false}         % Decimales con punto en vez de coma
\def\userphotosize {3.54}         % Dimensiones de la foto del usuario [cm]
\def\usersignaturesize {5.0}      % Ancho de la foto de la firma [cm]

% CONFIGURACIÓN DEL ESTILO DEL TEMPLATE
\def\dateseparator {--}           % Carácter que separa las fechas
\def\hfstyle {style1}             % Estilos de encabezado y pie de página (2 estilos)
\def\showuserphotoborder {true}   % Muestra un recuadro en la foto de perfil
\def\writelastchangeheader {true} % Escribe fecha último cambio en el encabezado
\def\writetitleheader {true}      % Escribe título en el encabezado

% CONFIGURACIÓN DE LOS COLORES DEL DOCUMENTO
\def\colorpage {white}            % Color de la página
\def\cvblocklinecolor {black}     % Color de las líneas divisorias de los bloques
\def\datecolor {black}            % Color de las fechas en entradas de instituciones
\def\headertextcolor {black}      % Color del texto en el header
\def\highlightcolor {yellow}      % Color del subrayado con \hl
\def\instentrytitlecolor {black}  % Color del título en entrada de instituciones
\def\instregioncolor {black}      % Color de la región en las instituciones
\def\lastchangeheadercolor {gray} % Color de fecha último cambio en el encabezado
\def\maintextcolor {black}        % Color principal del texto
\def\numcitecolor {black}         % Color del número de las referencias o citas
\def\otherentrytitlecolor {black} % Color título en entradas <otherentry>
\def\personaltblentcolor {black}  % Color de los títulos en la tabla de antecedentes
\def\showborderonlinks {false}    % Encierra los links por un recuadro de color
\def\tablelinecolor {black}       % Color de las líneas de las tablas
\def\titlecolor {black}           % Color de los títulos
\def\urlcolor {dark-blue}         % Color de los enlaces web y títulos de instituciones
\def\userphotobordercolor {black} % Color del borde de la foto del usuario

% ESTILO DE LOS TÍTULOS
\def\fontsizemaintitle {\huge}    % Tamaño título principal
\def\fontsizetitle {\Large}       % Tamaño título secciones
\def\stylemaintitle {\bfseries}   % Estilo título principal
\def\styletitle {\bfseries}       % Estilo título secciones

% MÁRGENES DE PÁGINA
\def\pagemarginbottom {1.9}       % Margen inferior página [cm]
\def\pagemarginleft {1.9}         % Margen izquierdo página [cm]
\def\pagemarginright {1.9}        % Margen derecho página [cm]
\def\pagemargintop {1.5}          % Margen superior página [cm]

% OPCIONES DEL PDF COMPILADO
\def\addindextobookmarks {false}  % Añade el índice a los marcadores del pdf
\def\cfgbookmarksopenlevel {1}    % Nivel de los marcadores (1: títulos)
\def\cfgpdfbookmarkopen {false}   % Expande marcadores hasta el nivel configurado
\def\cfgpdfcenterwindow {true}    % Centra la ventana del lector al abrir el pdf
\def\cfgpdfdisplaydoctitle {true} % Muestra nombre autor como título del pdf
\def\cfgpdffitwindow {false}      % Ajusta la ventana del lector al tamaño del pdf
\def\cfgpdfmenubar {true}         % Muestra el menú del lector al abrir el pdf
\def\cfgpdfpagemode {OneColumn}   % Modo de página en lector (OneColumn,SinglePage)
\def\cfgpdfpageview {FitH}        % Ajuste pag (Fit,FitH,FitV,FitR,FitB,FitBH,FitBV)
\def\cfgpdftoolbar {true}         % Muestra la barra de herramientas del lector pdf
\def\cfgshowbookmarkmenu {false}  % Muestra menú de los marcadores al abrir el pdf
\def\pdfcompileversion {7}        % Versión mínima del pdf a compilar (1.x)

% NOMBRES DE LOS OBJETOS
\def\headertitle {Currículum Vítae}   % Título principal en el header
\def\nomlastchange {Último cambio el} % Nombre apartado último cambio
\def\nompersonaldata {Antecedentes Personales} % Título de antecedentes personales
\def\nomsummary {Descripción}         % Título sección descripción
\def\present {presente}               % Fecha presente en <institutionitem>
