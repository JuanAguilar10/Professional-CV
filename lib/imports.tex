% Template:     Professional-CV
% Documento:    Importación de librerías
% Versión:      1.2.6 (22/10/2018)
% Codificación: UTF-8
%
% Autor: Pablo Pizarro R.
%        Facultad de Ciencias Físicas y Matemáticas
%        Universidad de Chile
%        pablo.pizarro@ing.uchile.cl, ppizarror.com
%
% Sitio web:    [http://latex.ppizarror.com/Professional-CV/]
% Licencia MIT: [https://opensource.org/licenses/MIT/]

% LIBRERÍAS IMPORTANTES
\usepackage[spanish,es-nosectiondot,es-lcroman]{babel} % Idioma
\usepackage[T1]{fontenc} % Caracteres acentuados
\usepackage{ifthen} % Manejo de condicionales

% LIBRERÍAS INDEPENDIENTES
\usepackage{array}         % Nuevas características a las tablas
\usepackage{booktabs}      % Permite manejar elementos visuales en tablas
\usepackage{color}         % Colores
\usepackage{colortbl}      % Administración de color en tablas
\usepackage{geometry}      % Dimensiones y geometría del documento
\usepackage{graphicx}      % Propiedades extra para los gráficos
\usepackage{lipsum}        % Permite crear textos dummy
\usepackage{mdwlist}       % Listas con encabezado
\usepackage{multicol}      % Múltiples columnas
\usepackage{relsize}       % Escalado avanzado
\usepackage{sectsty}       % Cambia el estilo de los títulos
\usepackage{selinput}      % Compatibilidad con acentos
\usepackage{setspace}      % Cambia el espacio entre líneas
\usepackage{soul}          % Permite subrayar texto
\usepackage{textcomp}      % Simbología común
\usepackage{url}           % Permite añadir enlaces
\usepackage{wasysym}       % Contiene caracteres misceláneos
\usepackage{wrapfig}       % Permite comprimir imágenes
\usepackage{xcolor}        % Administración de color avanzado
\usepackage{xspace}        % Adminsitra espacios en párrafos y líneas

% LIBRERÍAS CON PARÁMETROS
\usepackage[pdfencoding=auto,psdextra]{hyperref} % Enlaces, referencias
\usepackage[final]{pdfpages} % Permite administrar páginas en pdf

% LIBRERÍAS DEPENDIENTES
\usepackage{bookmark}      % Administración de marcadores en pdf
\usepackage{fancyhdr}      % Encabezados y pie de páginas
\usepackage{float}         % Administrador de posiciones de objetos
\usepackage{hyperxmp}      % Etiquetas opcionales para el pdf compilado
\usepackage{multirow}      % Agrega nuevas opciones a las tablas
